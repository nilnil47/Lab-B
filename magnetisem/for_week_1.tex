\documentclass{article}
\usepackage{amsmath}
\title{for week 1}
\author{Neil Dotan}
\date{\today}
\begin{document}
\maketitle
\section{faraday and lens law}
faraday law is maxwell equation number 4:
$$\vec{\nabla} \times \vec{E} = -\frac{\partial B}{\partial t} $$

\section{2}
The magnetic field inside a coil with length L and N loops is:
$$B = NL$$

\section{3}
there are 2 microscopic origins for the magnetic moment:
1. microscopic electric currents of free electrons
2. spins of electrons in the material atoms

\section{4}
Diamagnetism is a very weak form of magnetism that is induced by a change in the orbital motion of electrons due to an applied magnetic field.
in Diamagnetism the direction of the contribution to the magnetic field is in the opposite direction of the external field
Diamagnetism

Paramagnetism is a form of magnetism whereby some materials are weakly attracted by an externally applied magnetic field, and form internal, induced magnetic fields in the direction of the applied magnetic field.
in Paramagnetism the direction of the contribution to the magnetic field at the same direction of the external field
paramagnetism is possible du to the phenomenon that after appling external magnetic field,
the net magnetic moment cause by the spins of the atom is no longer zero, but rethere pointing to the direction of the external field

\section{}
Magnetic susceptibility is the ratio $\frac{M}{H}$ where $M$ is the magnetic moment per voulme
and H is the applied magnetic field. Diamagnetism have negative susceptibility and paramagnets
have positive susceptibility

permeability is the measure of magnetization that a material obtains in response to an applied magnetic field.
permeability is marked with $\mu$ and the $\mu_0$ is the permeability of the vacuum


\section{}
fromagnetic is magnet with different domain. each domain act like a magnetic dipole
when no external field is exists the domain are sorted randomly in the material so there is
no magnetization effect. when extrenal field exits the domain are sorted in the direction of the external
magnetic field and we get a relatively large magnetization effect.

anti feromagnets are materials like feromagnets, but with the different that on
effect of external magnetic field, some of the domains are sorted at the external field direction
and some sorted to the opposite direction of the external field.
therefore the effects are cancel each other. this make the magnetizem be weaker than
feromagnets without applied external field - where there the domains are sorted randomly

\section{}
kerry temperature is the temperature which the innear kennetic energy of the material is
high enought to destroy the fromagnet stracture

\section{}
the energy exhange energy is the phenomenon that makes feromagnetisem to happen.
on feromagnets the exchange energy is what makes the domain to be so strong magneticly relative to the other materials.
we thing that the phenomenon happen because a Wave interference of the electrons wave function
that order to the same direction

\section{}
$$B = \mu H$$
$$M = \chi H$$
where $\chi $  is the magnetic susceptibility

\section{}
magnetic domains are domains in the material that have magnetic dipole at the same direction

magnetic domain walls are the are the boundaries between the domain areas.
their sizes are usually 100 - 150 atoms\

\section{}
hesteries loop: https://reader.bookfusion.com/books/2643917-lecture-25

\section{}
magnetic Saturation is the the state of the pheromanget when
all the domain has ordered to the same direction as the external field


\section{}
magnetic curve is the graph M as function of H. to get this
we need to measue the M as we change H till we over on all the hesteries loop

\section{}
the difference berween the B of H and M of H is that on B H

\section{}
magnetic circles - to aks

\section{}
the area of the hysteresis loop is the work that have been done by the
H. this energy is release as heat

\end{document}

